\chapter{Embeddings}

\epigraph{
    Nets are for fish; Once you get the fish you can forget the net.\\
    Words are for meaning; Once you get the meaning you can forget the words.
}{Zhuangzi}

Introduzione

\section{Embeddings}
L’asfalto per cui Los Angeles è famosa si trova soprattutto sulle sue autostrade. 
Ma nel mezzo della città c’è un’altra distesa di asfalto, le pozze di catrame di La Brea, 
e questo asfalto conserva milioni di ossa fossili risalenti all’ultima delle ere glaciali dell’Epoca Pleistocenica. 
Uno di questi fossili è lo \textit{Smilodon}, o tigre dai denti a sciabola, immediatamente riconoscibile 
per i suoi lunghi canini. Circa cinque milioni di anni fa viveva un tipo completamente diverso di 
tigre dai denti a sciabola, 
chiamata \textit{Thylacosmilus}, in Argentina e in altre parti del Sud America. 
Thylacosmilus era un marsupiale, mentre Smilodon era un mammifero placentato, 
ma Thylacosmilus aveva gli stessi lunghi canini superiori e, come Smilodon, 
una protezione ossea (una flangia) sulla mandibola inferiore.
La somiglianza tra questi due mammiferi è uno dei numerosi esempi di evoluzione parallela o convergente, 
in cui particolari contesti o ambienti portano all’evoluzione di strutture molto simili in specie differenti (Gould, 1980).

\begin{figure}[h!]
    \centering
    \begin{subfigure}{0.45\textwidth}
        \centering
        \includegraphics[width=\linewidth]{pictures/smilodon_scheleton.jpg}
        \caption{Smilodon}
    \end{subfigure}
    \hfill
    \begin{subfigure}{0.45\textwidth}
        \centering
        \includegraphics[width=\linewidth]{pictures/Thylacosmilus_Holotype_FMNH.jpg}
        \caption{Thylacosmilus}
    \end{subfigure}
    \caption{Confronto tra Smilodon e Thylacosmilus}
\end{figure}


Il ruolo del contesto è importante anche per la somiglianza di un tipo di organismo meno biologico: 
la parola. Le parole che compaiono in contesti simili tendono ad avere significati simili. 
Questo legame tra somiglianza nella distribuzione delle parole e somiglianza nel loro significato è chiamato 
ipotesi distribuzionale. L’ipotesi fu formulata per la prima volta negli anni 
’50 da linguisti come Joos (1950), Harris (1954) e Firth (1957), che notarono che parole sinonime 
(come oculist ed eye-doctor) tendevano a comparire nello stesso ambiente (ad esempio vicino a parole come eye o examined), 
con la quantità di differenza di significato tra due parole “corrispondente più o meno alla quantità di differenza nei 
loro ambienti” (Harris, 1954, p. 157).


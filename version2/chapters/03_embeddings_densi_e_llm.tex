\chapter{Embeddings densi e Large Language Models}

\section{Embeddings testuali e rappresentazioni semantiche}
\subsection{Word, sentence e document embeddings}
\subsection{Embeddings nei modelli di linguaggio moderni}

\section{Proprietà degli embeddings densi}
\subsection{Alta dimensionalità e informazione distribuita}
\subsection{Continuità e non-monosemanticità}

\section{Il problema dell’interpretabilità}
\subsection{Limiti dell’interpretazione dimensionale}
\subsection{Conseguenze pratiche per controllo e spiegabilità}

\section{Superposition e concetti distribuiti}
\subsection{Sovrapposizione semantica negli embeddings}
\subsection{Relazione con la capacità del modello}

\section{La necessità di rappresentazioni sparse}
\subsection{Sparsità come principio induttivo}
\subsection{Disentanglement e interpretabilità}

\section{Sparse Autoencoders come strumento di disentanglement}
\subsection{Dagli embeddings densi alle feature sparse}
\subsection{Collegamento con il lavoro di tesi}
